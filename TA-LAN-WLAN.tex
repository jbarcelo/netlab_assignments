\chapter{Traffic Analysis, LAN, WLAN}

\section{Layered Networks}
Data networks are organized in layers.
These layers communicate with each other using interfaces.
Each layer offers some services to the layer on top of it.
Also, each layer in one end of the communication connects to the same layer at the other end of the communication.

In theory, the layers are relatively independent from each other.
A layer can see other layers as \emph{black boxes} without having to care about implementation details.
This approach is a common practice in engineering as it allows each team of engineers to focus in a particular problem which is just a piece of a puzzle to solve a bigger problem.

Each layer accomplishes some particular tasks.
For example, the physical layer transmits symbols containing information over a transmission medium such as an optical fiber or a radio channel.
The network layer is assigned the task of taking routing decisions to move packets from one network to the other.

The Open Systems Interconnection model (OSI) partitions a communication system in seven layers: physical, data link, network, transport, session, presentation and application.
The TCP/IP model (Transfer Control Protocol/Internet Protocol) is more specific to the Internet and differentiates five layers: physical, link, network, transport, and application.`

The physical layer deals with the transmission medium and transfers information symbols over it.
The link layer is in charge of one-hop communication.
The network layer is responsible for connecting multiple networks and therefore should be capable of routing data through multiple hops.
The transport network takes care of end-to-end communication between two hosts, and can multiplex different instances of communications in each host.
The application layer uses the end-to-end communication to offer some kind of service to the network user.

In practice, data is divided in chunks of information.
It can be convenient to give different names to the different chunks of information used by the different layers: symbols (PHY), frames (LINK), IP packet (NET), UDP datagram or TCP segment (TRA), and message (APP).

The application messages are encapsulated in TCP segments or UDP datagrams which, in turn, are encapsulated into IP packets that are finally encapsulated into link layer frames.
Regular users do not need to worry about all these data chunks and encapsulation and de-encapsulation processes.
However, for network engineers designing systems or debugging networks it is very useful to peek into the data chunks of all communications layer to find the errors.

\section{Address mapping and ARP}

At a higher level an application is identified by an IP address and a port number. 
In principle, the IP address identifies a unique host in the Internet and the port number identifies a particular application in that particular host.
The IP address has two different parts - network and host -  that can have variable size.
The actual size of each part is specified by the network mask.
As the IP address depends on the network that the host belongs to, it will change as the host roams from one network to the other.
The IP addresses are network layer addresses and are hierarchical as it can be used by routers to send packets to the right network.

Since some link layer technologies such as ethernet support multiple hosts, they also need a link layer address.
In the case of ethernet, such address is unique for each manufactured network interface card (NIC) and it is also known as a hardware address.
It is necessary to have some mechanism to translate from IP addresses to HW addresses.

The link layer receives IP packets from the network layer.
If the IP belongs to the same network, the packet will be sent directly to the destination.
Otherwise, the packet will be sent to the next router (for example the default gateway).
In a typical LAN/WLAN setting,  IP address of the next hop is provided by either the IP configuration or the routing table
In any case, the link layer needs to know the hardware address of the next hop.

To translate from an IP address to an ethernet address, the Address Resolution Protocol (ARP) is used.
For example, to find out the hardware address associated to IP 192.168.1.1 in an ethernet network, the inquiring host broadcasts a message to the whole network ``Who has IP 192.168.1.1?''.
The owner of the address in question answers with a directed message with its IP and its hardware address.

To avoid the sending of an ARP request for every transmitted packet, the ARP information is saved from some time in the ARP cache.
The link layer will first check the ARP cache and only send a broadcast message if the required information is not present in the cache.



The typical configuration of an end host includes the IP of the default gateway.
This is the first hop for packets destined to external networks.
When the link layer is encapsulating an IP packet needs to know the hardware address of the default gateway.
Similarly

