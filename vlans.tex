\chapter{Virtual LANs (VLANs)}

In this lab assignment the switches will be configured to create different VLANs.
The IPs of the switches are 192.168.1.110, 192.168.1.111 and 192.168.1.112.
An sketch on the blackboard specifies the switch to which your PC is connected.
Each VLAN has a unique identifier that takes values between 0 and 4094. 
In this assignment we will use the identifiers 10 and 20.


Each group will use three PCs.
One of the PCs is for managing the switch and needs an IP address of the same range as the IP of the switch (check the blackboard for the details).
The IPs of the other two switches must belong to the range of the VLAN that you are going to use (192.168.\{10,20\}.XX).

\section{Creation of a VLAN}

\begin{table}[!t]
%% increase table row spacing, adjust to taste
\renewcommand{\arraystretch}{1.3}
% if using array.sty, it might be a good idea to tweak the value of
%\extrarowheight as needed to properly center the text within the cells
\caption{Command modes}
\label{tab:modes}
\centering
\begin{tabular}{|c|p{5cm}|c|}
\hline
Command Mode & Access & Prompt\\
\hline
User EXEC & Connecting to the switch & Switch> \\
Privileged EXEC & Using the enable command in the ``User EXEC'' mode & Switch\# \\
Global Configuration & Using the ``configure terminal'' command in the ``Privileged EXEC'' mode& Switch(config)\# \\
        Interface Configuration & Using the ``interface $<$interface-name$>$'' command in the ``Global Configuration'' mode specifying the interface that we want to configure, e.g. FastEthernet0/4& Switch(config-if)\# \\
\hline
\end{tabular}
\end{table}


Table \ref{tab:modes} describes the four possible modes of interaction with the switch.
The first one offers limited information about the switch.
It is possible to find more detailed information in the second (privileged EXEC) mode.
The third mode allows the configuration of general aspects of the switch while and the last one is to configure a specific interface.
The commands available in each of the modes are different.
Make sure you are in the right mode before issuing a command.
It is possible to move to the previous mode using the ``exit'' command.

Use a telnet client to connect to the switch and observe which is the initial mode.
You can use the command ``?'' to obtain information about the possible commands in a given mode.
Additionally, you can also follow a partial command by ``?'' to obtain more information about how to use the command and the required parameters.
For example ``ip address ?'' would give you information about the parameters you could use after address.

Enter the mode \emph{privileged} EXEC and use the command \texttt{show running config} to see the current configuration of the switch.
Answer the following questions.
\emph{How many VLANs can you observe? (Note that this is not necessarily the number of VLANs in the switch).}
\emph{How many Fast Ethernet interfaces are available?}
\emph{What is the VLAN1 administrative address?}
\emph{What is the status of VLAN1?}

There exists a \emph{default} VLAN which has the number 1.
Use the command \texttt{show vlan}.
\emph{How many VLANs are there in the switch?}
For each of the interfaces identify the ID, the name, the status, the assigned ports and the type.
Include this information in the report.

Enter the config mode and try to delete the default VLAN: 
\texttt{Switch-B(config)#no vlan 1}.
\emph{What happens?}

In the config mode, use the ``?'' to find which commands can be used in this mode.
Use the \texttt{vlan <id>} to create a new command.














