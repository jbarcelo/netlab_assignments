\chapter{About the course}

\section{Course Data}

Code: 21728

Course name: ``Laboratori de Xarxes i Serveis''

Teachers: Ruizhi Liao, Alex Bikfalvi and Jaume Barcelo

Credits: 4

Year: 2nd year

Trimester: Spring

\section{Introduction}

The goal of this course is to acquire hands-on experience with networking equipment such as access points, switches, routers and firewalls.
The students are assumed to be familiar with the high-level functionality of each of these devices.
Nevertheless, the actual configuration of the equipment and the construction of prototype networks will provide further insights on the operation of network devices.
After the course, the student will be ready to plan and configure a small network.



\section{Syllabus}
\begin{itemize}
  \item Lectures
  \begin{enumerate}
    \item Introduction to the Networking Laboratory
    \item Traffic analysis and IEEE 802.11 WLANs
    \item Virtual Local Area Networks and Spanning Tree Protocol
    \item Routers
    \item Firewalls
  \end{enumerate}
\item Lab Assignments
  \begin{enumerate}
    \item Traffic analysis
    \item IEEE 802.11 Wireless Local Area Networks (WLANs)
    \item Virtual local area networks (VLANs)
    \item Spanning Tree Protocol (STP)
    \item Routing
    \item Firewalls
  \end{enumerate}
\end{itemize}

\section{Bibliography}

TBD

\section{Evaluation Criteria}

The grading is distributed as follows:
\begin{itemize}
\item Lab assignments, 70\%
\item Continuous assessment quiz, 10\%
\item Final exam, 20\%
\end{itemize}

It is necessary to obtain a decent mark (half of the available points)  in all the different evaluation aspects.

\section{Lab Report}
For each lab assignment, it is necessary to prepare a lab report answering all the questions. 
The students are also expected to include additional information, explanation and comments besides those explicitly asked in the assignment.

\section{Survival guide}

\subsection{Questions and doubts}
We like to receive questions and comments.
Normally, the best moment to express a doubt is during the class, as it is likely that many people in the class share the same doubt.
If you feel that you have a question that needs to be discussed privately, we can discuss it right after the class.

\subsection{Continuous feedback}
At the end of lectures, we will ask you to provide some feedback on the course. 
In particular, we always want to know:
\begin{itemize}
\item What is the most interesting thing we have seen in class.
\item What is the most confusing thing in the class.
\item Any other comment you may want to add.

\end{itemize}

\subsection{How to make you teachers happy}

Avoid speaking while we are talking.
