\chapter{Routing}

\section{Home preparation}
Two of the more used routing protocols are RIP and OSPF.
Find information about these two protocols and compare them.
Describe what is the format of a routing table and explain how each of the protocols work.

Read the following quick guide:
\url{www.jaumebarcelo.info/teaching/lxs/ipsec/routing/GUIA_RAPIDA_CISCO_2010.pdf}

\section{First Session}

In this first session each group will work with a router. 
The goals of this session are:
\begin{itemize}
\item Getting familiar with the configuration method.
\item Configuring ethernet interfaces.
\item Observe the RIP protocol in action.
\item Save the configuration in an external TFTP server.
\end{itemize}

Before disconnecting the computer from the Internet, download the TFTP server.
\url{http://tftpd32.jounin.net/}
Save it to one of the computers that you will use to connect to the router.

The routers are connected among them using the Ethernet interfaces and forming a topology that you will have to find out.
Use the console to connect to the routers (Hyperterminal or putty, 9600 bps, COM6).
The escape keystroke to exit a ping in a router is Ctrl-Alt-6.

\subsection{Checking the router status}

Use the console to connect to your router and explore the following commands.
Prepare a summary of what you can see with each command.
\begin{itemize}
\item show version
\item show protocols
\item show interfaces
\item show processes
\item show mem
\item show ip route
\item show history
\end{itemize}

\subsection{Create a ``Running'' and ``Startup'' configuration}
Enter the ``privileged EXEC'' mode (enable, pass ``cisco'') and global configuration (conf term). 
Find the commands to:
\begin{itemize}
\item show and change the router name
\item debugging mode configuration
\item send pings from the router
\item activate ``fair queueing'' at the ethernet interfaces (Fa0/0).
\end{itemize}

Use the command ``show running-config'' from the privileged mode.
Save the current configuration to the startup configuration (copy running-config startup-config).

\subsection{IP addresses configuration}
Go to your physical router box and check which interfaces are visible.

Use the ``show interfaces'' command to see what interfaces are available in the router.
Fill in a table that includes
\begin{itemize}
\item Interface name
\item MTU
\item Bandwidth
\item Encapsulation protocol
\end{itemize}

Enter the Ethernet interface configuration mode with the command 
\begin{lstlisting}
# interface <interface name>
\end{lstlisting}
and set the IP address to 192.168.10.XX, where XX is your group ID plus 10.
Use a /24 netmask.

\begin{center}
\sffamily\small
\begin{tabular}{>{\columncolor{tablegray}}p{15cm}}
\rowcolor{tableheader}
\multicolumn{1}{>{\columncolor{tableorange}}l}{Question}\\
What is the command that you have used?\\
\hline
\end{tabular}
\end{center}

Use the command
\begin{lstlisting}
# show interfaces
\end{lstlisting}
to verify the IP address assignment, and enable it with the command
\begin{lstlisting}
# no shutdown
\end{lstlisting}

Verify the line status and the interface status using the command 
\begin{lstlisting}
# show protocols
\end{lstlisting}

Use the commands
\begin{lstlisting}
# show cdp neighbors
\end{lstlisting}
and
\begin{lstlisting}
# show cdp neighbors detail
\end{lstlisting}
to see the neighboring devices.

Write down the information received from the different interfaces.
\begin{itemize}
\item neighbor identifier
\item ip address
\item port
\end{itemize}

Use the ping command to test the connectivity to the other routers in the lab and write down the round-trip times and other results that you may consider relevant.

Find out which is the network topology and sketch it in a figure.

Use the ``telnet'' command to connect to the router.
\begin{center}
\sffamily\small
\begin{tabular}{>{\columncolor{tablegray}}p{15cm}}
\rowcolor{tableheader}
\multicolumn{1}{>{\columncolor{tableorange}}l}{Question}\\
Is it possible to remotely configure a router?\\
\hline
Is login and password required?\\
\hline
Does a console user notice that there is an ongoing telnet connection?\\
\hline
Use telnet to change a parameter of the router (e.g., the name) and verify the changes both using the console and the telnet connection. What happens?\\
\hline
\end{tabular}
\end{center}

Do messages appear on the console when changes are done over telnet? What information is included in these messages?\\

Logout the telnet session to the Cisco router.

\subsection{IP routing configuration}

In this exercise we will enable the RIP protocol and check the status of the routing table as well as the RIP transactions of each router.

Check whether IP routing is enabled using the command
\begin{lstlisting}
# show protocols
\end{lstlisting}

\begin{center}
\sffamily\small
\begin{tabular}{>{\columncolor{tablegray}}p{15cm}}
\rowcolor{tableheader}
\multicolumn{1}{>{\columncolor{tableorange}}l}{Question}\\
What is the status of IP routing?\\
\hline
\end{tabular}
\end{center}

Enter the global config mode and enter the submenu router.
\begin{center}
\sffamily\small
\begin{tabular}{>{\columncolor{tablegray}}p{15cm}}
\rowcolor{tableheader}
\multicolumn{1}{>{\columncolor{tableorange}}l}{Question}\\
What is the purpose of this submenu?\\
\hline
\end{tabular}
\end{center}

Use the ``?'' command the available routing protocols and write down the results.
Enter into the configuration of RIP.

Use the command 
\begin{lstlisting}
# network <your network>
\end{lstlisting}
to associate your network to the RIP routing process.
Assume that we are working with ``C'' class addresses.
Therefore, the last byte of the network address must be 0.

Verify that the RIP protocol is now enabled and that your network has been recognized by the router using the command
\begin{lstlisting}
# show ip protocol
\end{lstlisting}
Observe the relevant parameters and answer the following questions:
\begin{center}
\sffamily\small
\begin{tabular}{>{\columncolor{tablegray}}p{15cm}}
\rowcolor{tableheader}
\multicolumn{1}{>{\columncolor{tableorange}}l}{Question}\\
What is the use of the timers?\\
\hline
What are their values?\\
\hline
Are they too small, or too large?\\
\hline
What happens if we change the values?\\
\hline
\end{tabular}
\end{center}

Verify the status of the routing table with the command
\begin{lstlisting}
# show ip route
\end{lstlisting}

\begin{center}
\sffamily\small
\begin{tabular}{>{\columncolor{tablegray}}p{15cm}}
\rowcolor{tableheader}
\multicolumn{1}{>{\columncolor{tableorange}}l}{Question}\\
What is the meaning of each of the fields in the table?\\
\hline
How can we check which are the networks to which RIP protocols is associated?\\
\hline
If there is no information, why?\\
\hline
\end{tabular}
\end{center}

Work together with another group to do this part.
If there is no other group ready, skip this exercise and move back to it when another group reaches this point.

Add an static route to the other group's network.
Use the ``ip route'' command from the configuration mode.
Explain what happens when you use ``traceroute'' to the other groups router (both interfaces).
Repeat the experiment after deleting the static route in one of the routers.
Explains what happens and why.

The command 
\begin{lstlisting}
# debug ip rip
\end{lstlisting}
shows the RIP messages that are sent and received by the router.
\begin{center}
\sffamily\small
\begin{tabular}{>{\columncolor{tablegray}}p{15cm}}
\rowcolor{tableheader}
\multicolumn{1}{>{\columncolor{tableorange}}l}{Question}\\
What are the source and destination of these packets?\\
\hline
What information do we obtain?\\
\hline
\end{tabular}
\end{center}

\subsection{Saving the router configuration in a TFTP server}

A convenient way to store a router's configuration is using TFTP.
We need to install the TFTP server in a PC with connectivity (ping connectivity) to the router.
Install the server and configure in which folder you want to save the router's configuration.

In the router, execute the command 
