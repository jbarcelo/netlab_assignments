\chapter{LAN and WLAN}

\section{Home exercise}

Connect to the web configuration interface of your home access point and find out:
\begin{itemize}
\item The name of the wireless network (SSID or ESSID).
\item Frequency channel.
\item PHY layer data rates.
\item Supported security protocols.
\item Possibility of QoS differentiation.
\end{itemize}

Do a survey and find the information of available wireless networks (name, channel, security settings).
You can use Netstumbler or the command ``sudo iwlist wlan1 scan''.

\section{Equipment}

Each group requires at least two PCs.
If possible, three PCs is better than two.
Boot one of the PCs in Windows and the other in Linux.
The hardware you are going to use is the Cisco Aironet 1200 access point.
The user guide can be downloaded here: \url{http://www.jaumebarcelo.info/teaching/lxs/wlan/WLAN_manual.pdf}
The firmware of the access point is CISCO IOS Version 12.3(8)JA2.

Install a FTP server in one of the computers (e.g., Filezilla in windows or yum install vsftpd in linux).
The web browser can be used as an FTP client.

If you decide to use Filezilla, connect locally from the same PC using the loopback interface 127.0.0.1 .
Create a user (user:test/pass:test) and share a folder with large files.

Remove the proxy configuration of the client computer.

\section{Basic WLAN configuration}

Interconnect the windows box and the linux box using a cross-over cable.
Check layer-2 connectivity using the LED or the mii-tool command in Linux.
Check layer-3 connectivity and measure round-trip-time using ping. 
Configure the interfaces if needed.
Estimate the available bandwidth using FTP transfer or iperf.
Change the speed to 10 Mbps and estimate the bandwidth again.


\emph{Is the maximum transmission speed reached? Why?}

\section{WLAN basic configuration}

WLANs can be used as an access point to LANs.
They can also be used to interconnect to LANs using WDS.

First connect the AP to the PC using Windows.
This can be either a direct connection or a connection using the patch panel.
You will find the AP's IP on a post-it, and the administrator user is ``Cisco'' and the password field must be left blank.

Use the express set-up to configure the AP.
AP Name: LABXARXES_GRUP_XX.
SSID: grupXX.
Channel: default.
Transmit power: default.

Make sure that the radio interface is up. 
Indicate what are the security options available.
Try different settings and configurations and then connect the AP to the laboratory switch.

Connect the WiFi interface to the Linux box and connect the computer to the AP that you have just configured.
Disable the wired interface to make sure that you are using the wireless interface.
Check that you have network connectivity and use the ``ifconfig'' command to look at the interface configuration.

If you have network connectivity, you should be able to ping the other computers of your group (the ones with wired connection) and also be able to connect to the Internet.

Take measures from the wireless computer to the wired one and the other way around.



